
\section{Restructuring of the conceptual schema}

\subsection{Restructured conceptual schema}\label{RestructuredConceptualSchema}
\vspace{12px}

\begin{figure}[H]
  \centerline{\includesvg[inkscapelatex=false,width=1.4\columnwidth]{images/restructured-conceptual-schema.svg}}
  \caption{Restructured Conceptual schema}
\end{figure}

% \begin{figure}[H]\label{RestructuredConceptualSchema}
%   \centering
%   \includegraphics[scale=0.26, angle=270]{RestructuredConceptualSchema.jpg}
%   \caption{Restructured Conceptual schema}
% \end{figure}

\pagebreak

\textbf{Notes on the restructured conceptual schema diagram (\ref{RestructuredConceptualSchema}):\\}
\begin{itemize}
  \item \textbf{Redundancy analysis:}
  \newline The schema has been designed to avoid redundancy. There are no redundant attributes or entities present.
  \item \textbf{Elimination of multi-valued attributes:}
 \newline The multi-valued attribute "kitColors" in the "Team" entity has been transformed into a new "KitColor" entity with the appropriate cardinalitites.
  \item \textbf{Elimination of composite attributes:}
  \newline In order to simplify the schema and reduce the load on operations, the composite attribute "Name" has been decomposed. Instead of creating a separate "Name" entity, we've directly incorporated its constituent attributes into the "Person" entity.
  \item \textbf{Elimination of ISA and generalization between entities:}
  \newline In the restructured schema, the ISA relationships that existed between the "Player", \newline "CoachingStaff", "Manager", and "Person" entities have been replaced. We've introduced corresponding binary relationships with appropriately defined cardinalities.
  \item \textbf{Choice of the primary identifiers of entities:}
  \newline The schema is designed in a way that assigns proper primary keys to the entities. For entities with more than one potential identifier, we've preferred using internal identifiers.
  \item \textbf{Choice of the primary identifiers of relationships:}
  \newline The identification of relationships in the schema is appropriately done. The primary key for a relationship is either a unique attribute (for unary relations) or a set of unique attributes (for binary and ternary relations). For instance, in the "Train" relationship, we've selected the combination of "coachingStaffID" and "playerID" as the primary key. This combination not only simplifies the schema but is also the correct choice for maintaining unique identification within this relationship.
\end{itemize}

\pagebreak

\subsection{Restructured data dictionary}

\vspace{12px}

{\centering \textbf{Restructured data dictionary: Entities}\\}

\begin{table}[H]
  \def\arraystretch{1.20}%  1 is the default, change whatever you need
  \centering
  \begin{tabular}{|>{\columncolor{myColor}}  m{3cm} | m{4cm}| m{3cm} | m{3cm} |}
    \hline
      \rowcolor{myColor}
    {\textcolor{white}{\large \textbf{Entity}}} & {\textcolor{white}{\large \textbf{Description}}} & {\textcolor{white}{\large \textbf{Attributes}}} & {\textcolor{white}{\large \textbf{Identifiers}}} \\
    \hline
     {\textcolor{white}{\textbf{Team}}} & Represents\phantom{,,}a\phantom{,,}football team & name\newline teamID\newline city\newline foundationYear & \{teamID\} \\
    \hline
    {\textcolor{white}{\textbf{Person}}}  & Represents a person involved in a football club & personID \newline dob \newline nationality\newline firstName,\newline middleName,\newline lastName & \{personID\} \\
    \hline
    {\textcolor{white}{\textbf{Player}}} & 	Represents a player in a team.
    & startingXI \newline appearances & \{personID\} \\
    \hline
    {\textcolor{white}{\textbf{CaptainHistory}}} & Represents a captain of a team.
    & startDate \newline endDate & \{startDate\} \\
    \hline
    {\textcolor{white}{\textbf{Contract}}} & Represents an agreement between a person and a team.
    & contractID\newline startDate\newline endDate\newline salary\newline jerseyNumber\newline position
    & \{contractID\} \\
    \hline
    {\textcolor{white}{\textbf{CoachingStaff}}} &  Represents a coaching staff member of a team 
    & role 
    & \{personID\} \\
    \hline
    {\textcolor{white}{\textbf{Manager}}} & Represents a manager of a team & yearsOfExperience & \{personID\} \\
    \hline
     {\textcolor{white}{\textbf{Sponsor}}} & Represents a sponsor for a team,or a player 
    & name \newline industry\newline sponsorID \newline industry \newline foundationYear & \{sponsorID\} \\
    \hline
     {\textcolor{white}{\textbf{T-Sponsorship}}} & Represents an agreement between a team and a sponsor 
     & tSponsorshipID\newline startDate\newline endDate\newline type 
     & \{tSponsorshipID, Sponsor\} \\
    \hline
   {\textcolor{white}{\textbf{P-Sponsorship}}} & Represents an agreement between a player and a sponsor 
     & pSponsorshipID\newline startDate\newline endDate\newline type 
     & \{pSponsorshipID, Sponsor\} \\
    \hline
     {\textcolor{white}{\textbf{KitColor}}} & Represents the kit colors of a team
     &  color 
     & \{color\} \\
    \hline
  \end{tabular}\label{tab:table8}
\end{table}


\pagebreak

{\centering \textbf{Restructured data dictionary: Relationships}\\}

\begin{table}[H]
  \def\arraystretch{1.25}%  1 is the default, change whatever you need
  \centering
  \begin{tabular}{|>{\columncolor{myColor}}  m{3.5cm} | m{4cm}| m{3cm} | m{2.5cm} | m{2.5cm} |}
    \hline
    \rowcolor{myColor}
   {\textcolor{white}{\large \textbf{Relationship}}} & {\textcolor{white}{\large \textbf{Description}}} & {\textcolor{white}{\large \textbf{Components}}} & {\textcolor{white}{\large \textbf{Attributes}}} & {\textcolor{white}{\large \textbf{Identifiers}}}  \\
    \hline
    {\textcolor{white}{\textbf{HasContract}}} & Connects a person with a contract. It represents the contract a person has.
    & Person,\newline Contract & & \{contractID, personID\} \\
    \hline
 {\textcolor{white}{\textbf{ContractWith}}} & Connects a Team with contracts. It represents the contracts a Team has.
    & Team,\newline Contract & & \{contractID, teamID\}\\
    \hline
   {\textcolor{white}{\textbf{HasTSponsorship}}} & Connects a team with its sponsorship agreement. It represents the sponsorship agreement a team has with a sponsor.
   & Team,\newline T-Sponsorship & & \{teamID,\newline tSponsorshipID\} \\
    \hline
    {\textcolor{white}{\textbf{HasPSponsorship}}} & Connects a player with its sponsorship agreement. It represents the sponsorship agreement a player has with a sponsor. 
   & Player,\newline P-Sponsorship & & \{playerID,\newline pSponsorshipID\} \\
    \hline
    {\textcolor{white}{\textbf{HasCaptainHistory}}} & Connects a team with its captain history records. It represents the history of captains for a team.
    & Team,\newline CaptainHistory &  & \{teamID\} \\
    \hline
    {\textcolor{white}{\textbf{HasCaptainRole}}} & Connects a player with their captain role. It represents the captain role of a player.
    & Player,\newline CaptainHistory & & \{playerID\}\\
    \hline
    {\textcolor{white}{\textbf{PlayerWithSponsor}}} & Connects a  P-Sponsorship with a sponsor. It represents the relationship between a sponsor and a sponsorship.
    & P-Sponsorship, \newline Sponsor &  & \{pSponsorshipID,\newline sponsorID\}   \\
    \hline
  \end{tabular}\label{tab:table4}
\end{table}

\pagebreak

{\centering \textbf{Restructured data dictionary: Relationships}\\}

\begin{table}[H]
  \def\arraystretch{1.25}%  1 is the default, change whatever you need
  \centering
  \begin{tabular}{|>{\columncolor{myColor}}  m{3.5cm} | m{4cm}| m{3cm} | m{2.5cm} | m{2.5cm} |}
    \hline
    \rowcolor{myColor}
   {\textcolor{white}{\large \textbf{Relationship}}} & {\textcolor{white}{\large \textbf{Description}}} & {\textcolor{white}{\large \textbf{Components}}} & {\textcolor{white}{\large \textbf{Attributes}}} & {\textcolor{white}{\large \textbf{Identifiers}}}  \\
    \hline
    {\textcolor{white}{\textbf{TeamWithSponsor}}} & Connects a T-Sponsorship  with a sponsor. It represents the relationship between a sponsor and a sponsorship.
    & T-Sponsorship, \newline Sponsor &  & \{tSponsorshipID,\newline sponsorID\} \\
    \hline
     {\textcolor{white}{\textbf{Trains}}} & Connects a coaching staff member with the players they train.
    & Coaching Staff,\newline Player &  & \{coachingStaffID,\newline playerID\}  \\
    \hline
     {\textcolor{white}{\textbf{ISA-P-P}}} & Connects a Person to a Player.
    & Person,\newline Player &  & \{personID\} \\
    \hline
     {\textcolor{white}{\textbf{ISA-CS-P}}} & Connects a Person to a Coaching Staff.
    & Coaching Staff,\newline Person &  & \{personID\} \\
    \hline
     {\textcolor{white}{\textbf{ISA-M-CS}}} & Connects a Coaching Staff to a Manager.
    & Coaching Staff,\newline Manager &  & \{coachingStaffID\}  \\
    \hline
     {\textcolor{white}{\textbf{HasKitColor}}} & Connects a Coaching Staff to a Manager.
    & Coaching Staff,\newline Manager &  & \{teamID,\newline color\}  \\
    \hline
  \end{tabular}\label{tab:table4}
\end{table}


\pagebreak

{\centering \textbf{Restructured data dictionary: External constraints}\\}

\begin{table}[H]
  \def\arraystretch{1.25}%  1 is the default, change whatever you need
  \centering
  \begin{tabular}{|>{\columncolor{myColor}}  m{1.5cm} | m{13.5cm}| }
    \hline
    \rowcolor{myColor}
    \multicolumn{2}{| c |}  {\textcolor{white}{\large \textbf{External Integrity Constraints}}} \\
    \hline
     {\textcolor{white}{\textbf{1}}} & A player cannot be a part of two different teams simultaneously. The start and end dates in the Contract should not overlap for the same player. \\
    \hline
     {\textcolor{white}{\textbf{2}}} & A person can have only one active Contract with a team at any given time. The end date of a Contract must be greater than its start date.\\
    \hline
     {\textcolor{white}{\textbf{3}}} & A team can have only one Captain at any given time. The start and end dates in the CaptainHistory should not overlap. \\
    \hline
     {\textcolor{white}{\textbf{4}}} & A player’s number of appearances cannot be negative.\\
    \hline
     {\textcolor{white}{\textbf{5}}} & A Person cannot hold the roles of Player and Coaching Staff concurrently. The start and end dates in the Contract should not overlap for the same person when they hold different roles\\
    \hline
     {\textcolor{white}{\textbf{6}}} &A Player’s jersey number should be within a predefined range (e.g., 1-99) \\
    \hline
     {\textcolor{white}{\textbf{7}}} & The start date of a Captain’s tenure (in CaptainHistory) cannot be in the future or earlier than the player’s contract start date with the team.\\
    \hline
    {\textcolor{white}{\textbf{8}}} & A Sponsor cannot sponsor the same team or player twice at the same time. The start and end dates in the T-Sponsorship or P-Sponsorship should not overlap for the same Sponsor \\
    \hline
  \end{tabular}\label{tab:table10}
\end{table}

\pagebreak

\subsection{Restructured table of volumes and operations}

\vspace{12px}

{\centering \textbf{Restructured table of volumes}\\}

\begin{table}[H]
  \def\arraystretch{1.25}%  1 is the default, change whatever you need
  \centering
  \begin{tabular}{|>{\columncolor{myColor}} m{4.5cm} | m{4.5cm}| m{4.5cm} |}
    \hline
    \rowcolor{myColor}
    {\textcolor{white}{\large \textbf{Concept}}} &  {\textcolor{white}{\large \textbf{Construct}}} &  {\textcolor{white}{\large \textbf{Volume}}} \\
    \hline
{\textcolor{white}{\textbf{Team}}} & Entity & 50 \\
\hline
{\textcolor{white}{\textbf{Person}}} & Entity & 1400 \\
\hline
{\textcolor{white}{\textbf{Player}}} & Entity & 1250 \\
\hline
{\textcolor{white}{\textbf{CoachingStaff}}} & Entity & 150 \\
\hline
{\textcolor{white}{\textbf{Manager}}} & Entity & 50 \\
\hline
{\textcolor{white}{\textbf{Sponsor}}} & Entity & 100 \\
\hline
{\textcolor{white}{\textbf{Contract}}} & Entity & 1400 \\
\hline
{\textcolor{white}{\textbf{CaptainHistory}}} & Entity & 50 \\
\hline
{\textcolor{white}{\textbf{T-Sponsorship}}} & Entity & 200 \\
\hline
{\textcolor{white}{\textbf{P-Sponsorship}}} & Entity & 300 \\
\hline
{\textcolor{white}{\textbf{KitColor}}} & Entity & 300 \\
\hline
{\textcolor{white}{\textbf{ContractWith}}} & Relationship & 1400 \\
\hline
{\textcolor{white}{\textbf{HasContract}}} & Relationship & 1400 \\
\hline
{\textcolor{white}{\textbf{Trains}}} & Relationship & 150 \\
\hline
{\textcolor{white}{\textbf{HasCaptainHistory}}} & Relationship & 50 \\
\hline
{\textcolor{white}{\textbf{HasCaptainRole}}} & Relationship & 50 \\
\hline
{\textcolor{white}{\textbf{HasPSponsorship}}} & Relationship & 1250 \\
\hline
{\textcolor{white}{\textbf{HasTSponsorship}}} & Relationship & 200 \\
\hline
{\textcolor{white}{\textbf{TeamWithSponsor}}} & Relationship & 200 \\
\hline
{\textcolor{white}{\textbf{PlayerWithSponsor}}} & Relationship & 300 \\
\hline
{\textcolor{white}{\textbf{ISA-P-P}}} & Relationship & 1250 \\
\hline
{\textcolor{white}{\textbf{ISA-CS-P}}} & Relationship & 150 \\
\hline
{\textcolor{white}{\textbf{ISA-M-CS}}} & Relationship & 50 \\
\hline
{\textcolor{white}{\textbf{HasKitColor}}} & Relationship & 150 \\
\hline
  \end{tabular}\label{tab:table11}
\end{table}
\pagebreak

\textbf{Operations of interest:}\label{TableOperations}
\begin{enumerate}
  \item Insert a new player to a team.
  \item Update team's coaching staff (including manager).
  \item Update player's contract.
  \item Assign a new captain.
  \item Insert a new sponsorship  (team or player).
  \item Retrieve team's players.
  \item Retrieve a player's contract.
  \item List all sponsorships for a player.
  \item List all sponsorships for a team.
  \item Transfer of a player to another team.
\end{enumerate}

\vspace{12px}

{\centering \textbf{Table of Operations}\\}

\begin{table}[H]
  \def\arraystretch{1.25}%  1 is the default, change whatever you need
  \centering
 \begin{tabular}{|>{\columncolor{myColor}} m{2cm} | m{3.5cm}| m{3.5cm} |}
    \hline
    \rowcolor{myColor}
    {\textcolor{white}{\large \textbf{Operation}}} &  {\textcolor{white}{\large \textbf{Type}}} &  {\textcolor{white}{\large \textbf{Frequency}}} \\
    \hline
    {\textcolor{white}{\textbf{1}}} & Interactive & 10/month  \\
    \hline
    {\textcolor{white}{\textbf{2}}} & Interactive & 3/month  \\
    \hline
   {\textcolor{white}{\textbf{3}}} & Interactive & 20/month  \\
    \hline
    {\textcolor{white}{\textbf{4}}} & Interactive & 1/year  \\
    \hline
    {\textcolor{white}{\textbf{5}}} & Interactive & 5/month  \\
    \hline
   {\textcolor{white}{\textbf{6}}} & Interactive & 100/day  \\
    \hline
    {\textcolor{white}{\textbf{7}}} & Interactive & 50/day  \\
    \hline
    {\textcolor{white}{\textbf{8}}} & Batch & 10/month  \\
    \hline
    {\textcolor{white}{\textbf{9}}} & Batch & 1/week  \\
    \hline
    {\textcolor{white}{\textbf{10}}} & Interactive & 2/year  \\
    \hline
  \end{tabular}\label{tab:table7}
\end{table}

\pagebreak

\subsection{Access tables}

We assume that a write access costs two read accesses in the total cost evaluation.

\vspace{12px}

{\centering \textbf{Access table for Operation 1}\\}
\begin{table}[H]
  \def\arraystretch{1.10}%  1 is the default, change whatever you need
  \centering
  \begin{tabular}{|>{\columncolor{myColor}} m{4cm} | m{4cm}| m{3cm} | m{2cm} |}
    \hline
    \rowcolor{myColor}
    {\textcolor{white}{\large \textbf{Concept}}} 
    &  {\textcolor{white}{\large \textbf{Construct}}} 
    &  {\textcolor{white}{\large \textbf{Accesses}}} 
    &  {\textcolor{white}{\large \textbf{Type}}}\\
    \hline
   {\textcolor{white}{\textbf{Person}}} & 	Entity & 1 & W \\
    \hline
    {\textcolor{white}{\textbf{Player}}} & 	Entity & 1 & W \\
    \hline
    {\textcolor{white}{\textbf{Contract}}} & Entity & 1 & W \\
    \hline
    {\textcolor{white}{\textbf{HasContract}}} & Relationship & 1 & W \\
    \hline
    {\textcolor{white}{\textbf{ContractWith}}} & Relationship & 1 & W \\
    \hline
     {\textcolor{white}{\textbf{ISA-P-P}}} & Relationship & 1 & W \\
    \hline
  \end{tabular}
  % * \small{We assume that a customer has normally between 1 and 2 phone numbers, thus 1.5 on average}
\end{table}
Total: 6*10 write accesses = 120 accesses per month

\vspace{12px}

{\centering \textbf{Access table for Operation 2a (Coaching Staff)}\\}
\begin{table}[H]
  \def\arraystretch{1.10}%  1 is the default, change whatever you need
  \centering
  \begin{tabular}{|>{\columncolor{myColor}} m{4cm} | m{4cm}| m{3cm} | m{2cm} |}
    \hline
    \rowcolor{myColor}
    {\textcolor{white}{\large \textbf{Concept}}} 
    &  {\textcolor{white}{\large \textbf{Construct}}} 
    &  {\textcolor{white}{\large \textbf{Accesses}}} 
    &  {\textcolor{white}{\large \textbf{Type}}}\\
    \hline
    {\textcolor{white}{\textbf{Person}}} & 	Entity & 1 & W \\
    \hline
   {\textcolor{white}{\textbf{CoachingStaff}}} & 	Entity & 1 & W \\
    \hline
    {\textcolor{white}{\textbf{ISA-CS-P}}} & 	Entity & 1 & W \\
    \hline
  \end{tabular}
  % * \small{We assume that a customer has normally between 1 and 2 phone numbers, thus 1.5 on average}
\end{table}
Total: 3*3 write accesses = 18 accesses per month

\vspace{12px}

{\centering \textbf{Access table for Operation 2b (Manager)}\\}
\begin{table}[H]
  \def\arraystretch{1.10}%  1 is the default, change whatever you need
  \centering
  \begin{tabular}{|>{\columncolor{myColor}} m{4cm} | m{4cm}| m{3cm} | m{2cm} |}
    \hline
    \rowcolor{myColor}
    {\textcolor{white}{\large \textbf{Concept}}} 
    &  {\textcolor{white}{\large \textbf{Construct}}} 
    &  {\textcolor{white}{\large \textbf{Accesses}}} 
    &  {\textcolor{white}{\large \textbf{Type}}}\\
    \hline
     {\textcolor{white}{\textbf{Person}}} & 	Entity & 1 & W \\
    \hline
     {\textcolor{white}{\textbf{Coaching Staff}}} & Entity & 1 & W \\
    \hline
   {\textcolor{white}{\textbf{Manager}}} & Entity & 1 & W \\
    \hline
    {\textcolor{white}{\textbf{ISA-M-CS}}} & Entity & 1 & W \\
    \hline
    {\textcolor{white}{\textbf{ISA-CS-P}}} & Entity & 1 & W \\
    \hline
  \end{tabular}
  % * \small{We assume that a customer has normally between 1 and 2 phone numbers, thus 1.5 on average}
\end{table}
Total: 5*3 write accesses = 30 accesses per month

\pagebreak

{\centering \textbf{Access table for Operation 3}\\}
\begin{table}[H]
  \def\arraystretch{1.10}%  1 is the default, change whatever you need
  \centering
  \begin{tabular}{|>{\columncolor{myColor}} m{4cm} | m{4cm}| m{3cm} | m{2cm} |}
    \hline
    \rowcolor{myColor}
    {\textcolor{white}{\large \textbf{Concept}}} 
    &  {\textcolor{white}{\large \textbf{Construct}}} 
    &  {\textcolor{white}{\large \textbf{Accesses}}} 
    &  {\textcolor{white}{\large \textbf{Type}}}\\
    \hline
  \textcolor{white}{\textbf{Person}} & Entity & 1 & R \\
\hline
\textcolor{white}{\textbf{Player}} & Entity & 1 & R \\
\hline
\textcolor{white}{\textbf{Player}} & Entity & 1 & W \\
\hline
\textcolor{white}{\textbf{Contract}} & Entity & 1 & R \\
\hline
\textcolor{white}{\textbf{Contract}} & Entity & 1 & W \\
\hline
\textcolor{white}{\textbf{HasContract}} & Relationship & 1 & W \\
\hline
\textcolor{white}{\textbf{ContractWith}} & Relationship & 1 & W \\
\hline
\textcolor{white}{\textbf{ISA-P-P}} & Relationship & 1 & R \\
\hline
  \end{tabular}
  % * \small{We assume that a customer has normally between 1 and 2 phone numbers, thus 1.5 on average}
\end{table}
Total: 4 write accesses + 4 read accesses = 12 accesses per operation
\newline Total accesses per month: 12 accesses * 20 operations/month = 240 accesses per month

\vspace{12px}

{\centering \textbf{Access table for Operation 4}\\}
\begin{table}[H]
  \def\arraystretch{1.10}%  1 is the default, change whatever you need
  \centering
  \begin{tabular}{|>{\columncolor{myColor}} m{4cm} | m{4cm}| m{3cm} | m{2cm} |}
    \hline
    \rowcolor{myColor}
    {\textcolor{white}{\large \textbf{Concept}}} 
    &  {\textcolor{white}{\large \textbf{Construct}}} 
    &  {\textcolor{white}{\large \textbf{Accesses}}} 
    &  {\textcolor{white}{\large \textbf{Type}}}\\
    \hline
   {\textcolor{white}{\textbf{Person}}} & Entity & 1 & R \\
    \hline
    {\textcolor{white}{\textbf{Player}}} & Entity & 1 & R \\
    \hline
    {\textcolor{white}{\textbf{Contract}}} & Entity & 1 & R \\
    \hline
     {\textcolor{white}{\textbf{HasContract}}} & Relationship & 1 & R \\
    \hline
     {\textcolor{white}{\textbf{CaptainHistory}}} & Entity & 1 & W \\
    \hline
     {\textcolor{white}{\textbf{HasCaptainRole}}} & Relationship & 1 & W \\
    \hline
     {\textcolor{white}{\textbf{HasCaptainHistory}}} & Relationship & 1 & W \\
    \hline
  \end{tabular}
  % * \small{We assume that a customer has normally between 1 and 2 phone numbers, thus 1.5 on average}
\end{table}
Total: 6 read accesses + 3 write accesses = 12 accesses
\newline Total accesses per year: 12 accesses * 1 = 12 accesses/year

\vspace{12px}

{\centering \textbf{Access table for Operation 5a (Team sponsorship)}\\}
\begin{table}[H]
  \def\arraystretch{1.10}%  1 is the default, change whatever you need
  \centering
  \begin{tabular}{|>{\columncolor{myColor}} m{4cm} | m{4cm}| m{3cm} | m{2cm} |}
    \hline
    \rowcolor{myColor}
    {\textcolor{white}{\large \textbf{Concept}}} 
    &  {\textcolor{white}{\large \textbf{Construct}}} 
    &  {\textcolor{white}{\large \textbf{Accesses}}} 
    &  {\textcolor{white}{\large \textbf{Type}}}\\
    \hline
   {\textcolor{white}{\textbf{Sponsor}}} & Entity & 1 & R \\
    \hline
    {\textcolor{white}{\textbf{Sponsor}}} & Entity & 1 & W \\
    \hline
    {\textcolor{white}{\textbf{T-Sponsorship}}} & Entity & 1 & W \\
    \hline
    {\textcolor{white}{\textbf{HasTSponsorship}}} & Entity & 1 & W \\
    \hline
     {\textcolor{white}{\textbf{TeamWithSponsor}}} & Relationship & 1 & W \\
    \hline
     {\textcolor{white}{\textbf{Team}}} & Entity & 1 & R \\
    \hline
  \end{tabular}
  % * \small{We assume that a customer has normally between 1 and 2 phone numbers, thus 1.5 on average}
\end{table}
Total: 4 read accesses + 3 write accesses = 10 accesses 
\newline Total accesses per month: 10 accesses * 5 operations/month = 50 accesses/month

\pagebreak

{\centering \textbf{Access table for Operation 5b (Player sponsorship)}\\}
\begin{table}[H]
  \def\arraystretch{1.10}%  1 is the default, change whatever you need
  \centering
  \begin{tabular}{|>{\columncolor{myColor}} m{4cm} | m{4cm}| m{3cm} | m{2cm} |}
    \hline
    \rowcolor{myColor}
    {\textcolor{white}{\large \textbf{Concept}}} 
    &  {\textcolor{white}{\large \textbf{Construct}}} 
    &  {\textcolor{white}{\large \textbf{Accesses}}} 
    &  {\textcolor{white}{\large \textbf{Type}}}\\
    \hline
   {\textcolor{white}{\textbf{Sponsor}}} & Entity & 1 & R \\
    \hline
    {\textcolor{white}{\textbf{Sponsor}}} & Entity & 1 & W \\
    \hline
    {\textcolor{white}{\textbf{P-Sponsorship}}} & Entity & 1 & W \\
    \hline
    {\textcolor{white}{\textbf{HasPSponsorship}}} & Entity & 1 & W \\
    \hline
     {\textcolor{white}{\textbf{PlayerWithSponsor}}} & Relationship & 1 & W \\
    \hline
     {\textcolor{white}{\textbf{Player}}} & Entity & 1 & R \\
    \hline
  \end{tabular}
  % * \small{We assume that a customer has normally between 1 and 2 phone numbers, thus 1.5 on average}
\end{table}
Total: 4 read accesses + 3 write accesses = 10 accesses 
\newline Total accesses per month:  10 accesses * 5 operations/month = 50 accesses/month

\vspace{12px}

{\centering \textbf{Access table for Operation 6}\\}
\begin{table}[H]
  \def\arraystretch{1.10}%  1 is the default, change whatever you need
  \centering
  \begin{tabular}{|>{\columncolor{myColor}} m{4cm} | m{4cm}| m{3cm} | m{2cm} |}
    \hline
    \rowcolor{myColor}
    {\textcolor{white}{\large \textbf{Concept}}} 
    &  {\textcolor{white}{\large \textbf{Construct}}} 
    &  {\textcolor{white}{\large \textbf{Accesses}}} 
    &  {\textcolor{white}{\large \textbf{Type}}}\\
    \hline
   {\textcolor{white}{\textbf{Team}}} & Entity & 1 & R \\
    \hline
    {\textcolor{white}{\textbf{ContractWith}}} & Relationship & 1 & R \\
    \hline
    {\textcolor{white}{\textbf{Contract}}} & Entity & 25* & R \\
    \hline
    {\textcolor{white}{\textbf{Player}}} & Entity & 25* & R \\
    \hline
     {\textcolor{white}{\textbf{Person}}} & Entity & 25* & R \\
    \hline
     {\textcolor{white}{\textbf{ISA-P-P}}} & Relationship & 25* & R \\
    \hline
  \end{tabular}
  * \small{We assume that an average team has 25 players.}
\end{table}
Total: 2 + 4*25 read accesses per operation = 102 accesses 
\newline Total accesses per day: 102 accesses * 100 operations/day = 10,200 accesses per day
\vspace{12px}

{\centering \textbf{Access table for Operation 7}\\}
\begin{table}[H]
  \def\arraystretch{1.10}%  1 is the default, change whatever you need
  \centering
  \begin{tabular}{|>{\columncolor{myColor}} m{4cm} | m{4cm}| m{3cm} | m{2cm} |}
    \hline
    \rowcolor{myColor}
    {\textcolor{white}{\large \textbf{Concept}}} 
    &  {\textcolor{white}{\large \textbf{Construct}}} 
    &  {\textcolor{white}{\large \textbf{Accesses}}} 
    &  {\textcolor{white}{\large \textbf{Type}}}\\
    \hline
   {\textcolor{white}{\textbf{Player}}} & Entity & 1 & R \\
    \hline
    {\textcolor{white}{\textbf{Person}}} & Entity & 1 & R \\
    \hline
    {\textcolor{white}{\textbf{ISA-P-P}}} & Relationship & 1 & R \\
    \hline
    {\textcolor{white}{\textbf{HasContract}}} & Relationship & 1 & R \\
    \hline
     {\textcolor{white}{\textbf{Contract}}} & Entity & 1 & R \\
    \hline
     {\textcolor{white}{\textbf{ContractWith}}} & Relationship & 1 & R \\
    \hline
      {\textcolor{white}{\textbf{Team}}} & Entity & 1 & R \\
    \hline
  \end{tabular}
  % * \small{We assume that an average team has 25 players.}
\end{table}
Total: 7 accesses * 50 operations/day = 350 accesses per day.

\pagebreak

{\centering \textbf{Access table for Operation 8}\\}
\begin{table}[H]
  \def\arraystretch{1.10}%  1 is the default, change whatever you need
  \centering
  \begin{tabular}{|>{\columncolor{myColor}} m{4cm} | m{4cm}| m{3cm} | m{2cm} |}
    \hline
    \rowcolor{myColor}
    {\textcolor{white}{\large \textbf{Concept}}} 
    &  {\textcolor{white}{\large \textbf{Construct}}} 
    &  {\textcolor{white}{\large \textbf{Accesses}}} 
    &  {\textcolor{white}{\large \textbf{Type}}}\\
    \hline
   {\textcolor{white}{\textbf{Player}}} & Entity & 1 & R \\
    \hline
    {\textcolor{white}{\textbf{Person}}} & Entity & 1 & R \\
    \hline
    {\textcolor{white}{\textbf{ISA-P-P}}} & Relationship & 1 & R \\
    \hline
    {\textcolor{white}{\textbf{HasPSponsorship}}} & Relationship & 1 & R \\
    \hline
     {\textcolor{white}{\textbf{P-Sponsorship}}} & Entity & 3* & R \\
    \hline
     {\textcolor{white}{\textbf{PlayerWithSponsor}}} & Relationship & 3* & R \\
    \hline
      {\textcolor{white}{\textbf{Sponsor}}} & Entity & 3* & R \\
    \hline
  \end{tabular}
  * \small{We assume that a Player has an average of 3 sponsorships.}
\end{table}
Total: 4 read accesses + 3*3 read accesses = 13 accesses 
\newline Total accesses per month: 13 accesses * 10 operations/month = 130 accesses per month.

\vspace{12px}

{\centering \textbf{Access table for Operation 9}\\}
\begin{table}[H]
  \def\arraystretch{1.10}%  1 is the default, change whatever you need
  \centering
  \begin{tabular}{|>{\columncolor{myColor}} m{4cm} | m{4cm}| m{3cm} | m{2cm} |}
    \hline
    \rowcolor{myColor}
    {\textcolor{white}{\large \textbf{Concept}}} 
    &  {\textcolor{white}{\large \textbf{Construct}}} 
    &  {\textcolor{white}{\large \textbf{Accesses}}} 
    &  {\textcolor{white}{\large \textbf{Type}}}\\
    \hline
   {\textcolor{white}{\textbf{Team}}} & Entity & 1 & R \\
    \hline
    {\textcolor{white}{\textbf{HasTSponsorship}}} & Relationship & 4* & R \\
    \hline
    {\textcolor{white}{\textbf{T-Sponsorship}}} & Entity & 4* & R \\
    \hline
    {\textcolor{white}{\textbf{TeamWithSponsor}}} & Relationship & 4* & R \\
    \hline
     {\textcolor{white}{\textbf{Sponsor}}} & Entity & 4* & R \\
    \hline
  \end{tabular}
  * \small{We assume that a Team has an average of 4 sponsorships.}
\end{table}
Total: 1 read access + 4*4 read accesses = 17 accesses
\newline Total accesses per week: 17 accesses * 1 operation/month = 17 accesses per week.

\vspace{12px}

{\centering \textbf{Access table for Operation 10}\\}
\begin{table}[H]
  \def\arraystretch{1.10}%  1 is the default, change whatever you need
  \centering
  \begin{tabular}{|>{\columncolor{myColor}} m{4cm} | m{4cm}| m{3cm} | m{2cm} |}
    \hline
    \rowcolor{myColor}
    {\textcolor{white}{\large \textbf{Concept}}} 
    &  {\textcolor{white}{\large \textbf{Construct}}} 
    &  {\textcolor{white}{\large \textbf{Accesses}}} 
    &  {\textcolor{white}{\large \textbf{Type}}}\\
    \hline
   {\textcolor{white}{\textbf{Contract}}} & Entity & 1 & R \\
    \hline
     {\textcolor{white}{\textbf{Contract}}} & Entity & 1 & W \\
    \hline
    {\textcolor{white}{\textbf{Player}}} & Entity & 1 & R \\
    \hline
     {\textcolor{white}{\textbf{Player}}} & Entity & 1 & W \\
    \hline
    {\textcolor{white}{\textbf{Team}}} & Entity & 1 & R \\
    \hline
     {\textcolor{white}{\textbf{Person}}} & Entity & 1 & R \\
    \hline
     {\textcolor{white}{\textbf{ISA-P-P}}} & Relationship & 1 & R \\
    \hline
    {\textcolor{white}{\textbf{ContractWith}}} & Relationship & 1 & W \\
    \hline
    {\textcolor{white}{\textbf{HasContract}}} & Relationship & 1 & W \\
    \hline
  \end{tabular}
  % * \small{We assume that a Team has an average of 4 sponsorships.}
\end{table}
Total: 5 read accesses + 4 write accesses = 13 accesses
\newline Total accesses per year: 13 accesses * 2 operations/year = 26  accesses per year.

\pagebreak