
\section{Restructuring of the relational schema}

\subsection{Restructured relational schema}

\vspace{12px}

{\color{ForestGreen}SnowstopProduct(\underline{Code},Name,Material,Color*,Price)}\\
{\color{Orange}\hspace{2mm} generalization constraint: {\color{Magenta} SnowstopProduct[Code] $\subseteq $ SnowRetainer[Code] $\cup$ RetainerHolder[Code] $\cup$ }} \\
{{\color{Magenta}\hspace{39mm} RetainerAccessory[Code] }} \\
{\color{Orange}\hspace{2mm} constraint: {\color{Magenta}Material is 'Zink Steel' or 'Stainless Steel' or 'Painted Steel' or 'Aluminium' or 'Copper'}} \\
{\color{Orange}\hspace{2mm} constraint: {\color{Magenta}Color is NULL if and only if Material is not 'Painted Steeel'}} \\

{\color{ForestGreen}SnowRetainer(\underline{Code},LinearResistance,RetainerType,Measure,Profile*)}\\
{\color{Orange}\hspace{2mm} foreign key: {\color{Magenta}SnowRetainer[Code] $\subseteq$ SnowstopProduct[Code]}} \\
{\color{Orange}\hspace{2mm} generalization constraint: {\color{Magenta} SnowRetainer[Code] $\cap $ RetainerHolder[Code] $\cap$ RetainerAccessory[Code] = $\varnothing $}} \\
{\color{Orange}\hspace{2mm} constraint: {\color{Magenta}RetainerType is 'Grid' or 'Tube'}} \\
{\color{Orange}\hspace{2mm} constraint: {\color{Magenta}Profile is NULL if RetainerType is 'Tube', otherwise Profile is not NULL}} \\

{\color{ForestGreen}RetainerHolder(\underline{Code},Resistance,RoofType,RetainerType)}\\
{\hspace{15mm}{\color{Orange}\hspace{2mm} foreign key: {\color{Magenta}RetainerHolder[Code] $\subseteq$ SnowstopProduct[Code]}} \\
{\color{Orange}\hspace{2mm} generalization constraint: {\color{Magenta} RetainerHolder[Code] $\cap $ SnowRetainer[Code] $\cap$ RetainerAccessory[Code] = $\varnothing $}} \\
{\color{Orange}\hspace{2mm} constraint: {\color{Magenta}Rooftype is 'Concrete Tile' or 'Ondulated Plate' or 'Trapezoidal Sheet' or}} \\
{\color{Magenta}\hspace{19.5mm}'Standing Seam Sheet' or 'Flat Tile'}\\
{\color{Orange}\hspace{2mm} constraint: {\color{Magenta}RetainerType is 'Grid' or 'Tube'}} \\

{\color{ForestGreen}RetainerAccessory(\underline{Code},Measure,Type,RetainerType)}\\
{\color{Orange}\hspace{2mm} foreign key: {\color{Magenta}RetainerAccessory[Code] $\subseteq$ SnowstopProduct[Code]}} \\
{\color{Orange}\hspace{2mm} generalization constraint: {\color{Magenta} RetainerAccessory[Code] $\cap $ SnowRetainer[Code] $\cap$ RetainerHolder[Code] = $\varnothing $}} \\
{\color{Orange}\hspace{2mm} constraint: {\color{Magenta}Type is 'Connection' or 'Ice Retainer'}} \\
{\color{Orange}\hspace{2mm} constraint: {\color{Magenta}RetainerType is 'Grid' or 'Tube'}} \\

{\color{ForestGreen}SupplyOffer(\underline{Code},ComputationCode,Date,TotalPrice,TotalResistance,Rows,Distance)}\\
{\color{Orange}\hspace{2mm} foreign key: {\color{Magenta}SupplyOffer[ComputationCode] $\subseteq$ SnowloadComputation[Code]}} \\
{\color{Orange}\hspace{2mm} inclusion: {\color{Magenta}SupplyOffer[Code] $\subseteq$ For-Customer[OfferCode]}} \\

{\color{ForestGreen}For-Customer(\underline{OfferCode,CustomerCode})}\\
{\color{Orange}\hspace{2mm} foreign key: {\color{Magenta}For-Customer[OfferCode] $\subseteq$ SupplyOffer[Code]}} \\
{\color{Orange}\hspace{2mm} foreign key: {\color{Magenta}For-Customer[CustomerCode] $\subseteq$ Customer[Code]}} \\

{\color{ForestGreen}Offer-Prod(\underline{OfferCode,ProductCode},Quantity)}\\
{\color{Orange}\hspace{2mm} foreign key: {\color{Magenta}Offer-Prod[OfferCode] $\subseteq$ SupplyOffer[Code]}} \\
{\color{Orange}\hspace{2mm} foreign key: {\color{Magenta}Offer-Prod[ProductCode] $\subseteq$ SnowstopProduct[Code]}} \\

{\color{ForestGreen}Customer(\underline{Code},Name,Zip,City,Discount)}\\
{\color{Orange}\hspace{2mm} foreign key: {\color{Magenta}Customer[Zip,City] $\subseteq$ City[Zip,Name]}} \\
{\color{Orange}\hspace{2mm} inclusion: {\color{Magenta}Customer[Code] $\subseteq$ Phone[CustomerCode]} \\
{\color{Orange}\hspace{2mm} constraint: {\color{Magenta}Discount $\geqslant$  0 and $\leqslant$ 30}} \\

{\color{ForestGreen}Phone(\underline{Number},CustomerCode)}\\
{\color{Orange}\hspace{2mm} foreign key: {\color{Magenta}Phone[CustomerCode] $\subseteq$ Customer[Code]}} \\

{\color{ForestGreen}SnowloadComputation(\underline{Code},Date,GroundLoad,RoofLoad)}\\
{\color{Orange}\hspace{2mm} foreign key: {\color{Magenta}SnowloadComputation[Code] $\subseteq$ BuildingSite[ComputationCode]}} \\

\pagebreak

{\color{ForestGreen}BuildingSite(\underline{Name,Zip,City},ComputationCode,Length,Width,Steepness,Covering)}\\
{\color{Orange}\hspace{2mm} foreign key: {\color{Magenta}BuildingSite[Zip,City] $\subseteq$ City[Zip,Name]}} \\
{\color{Orange}\hspace{2mm} foreign key: {\color{Magenta}BuildingSite[ComputationCode] $\subseteq$ SnowloadComputation[Code]}} \\
{\color{Orange}\hspace{2mm} constraint: {\color{Magenta}Covering is 'Concrete Tile' or 'Ondulated Plate' or 'Trapezoidal Sheet' or}} \\
{\color{Magenta}\hspace{19.5mm}'Standing Seam Sheet' or 'Flat Tile'}}\\
{\color{Orange}\hspace{2mm} key: {\color{Magenta}ComputationCode}} \\

{\color{ForestGreen}City(\underline{Zip,Name},Province,Altitude)}\\
{\color{Orange}\hspace{2mm} foreign key: {\color{Magenta}City[Province] $\subseteq$ Province[Shorthand]}} \\

{\color{ForestGreen}Province(\underline{Shorthand},Name,Zone,BaseLoad)}\\
{\color{Orange}\hspace{2mm} constraint: {\color{Magenta}Zone is 'I-A' or 'I-M' or 'II' or 'III}} \\
{\color{Orange}\hspace{2mm} constraint: {\color{Magenta}BaseLoad $>$ 0}} \\

\vspace{12px}

\begin{table}[H]
  \def\arraystretch{1.25}%  1 is the default, change whatever you need
  \centering
  \begin{tabular}{ | m{1.5cm} | m{13.5cm}| }
    \hline
    \multicolumn{2}{| c |} {\textbf{\large External integrity constraints in terms of the restructured relational schema}} \\
    \hline
    \color[HTML]{3531FF} \textbf{1} & The TotalResistance attribute of SupplyOffer must be higher than the RoofLoad attribute of its associated SnowloadComputation \\
    \hline
    \color[HTML]{3531FF} \textbf{2a} & The RetainerType attribute of RetainerHolder and SnowRetainer associated to the same SupplyOffer through the Offer-Prod relationship must have the same value \\
    \hline
    \color[HTML]{3531FF} \textbf{2b} & The RetainerType attribute of RetainerAccessory and SnowRetainer associated to the same SupplyOffer through the Offer-Prod relationship must have the same value \\
    \hline
    \color[HTML]{3531FF} \textbf{3} & The RoofType attribute of RetainerHolder associated to a SupplyOffer must have the same value as the Covering attribute of the BuildingSite associated to it following the relationships path by passing through SnowloadComputation entity.  \\
    \hline
    \color[HTML]{3531FF} \textbf{4} & The Code of SupplyOffer must participate between 2 and 4 times to the Offer-Prod relationship. In addition, the same Code being present as an instance in the Offer-Prod relationship must be associated with exactly one RetainerHolder, one SnowRetainer and between zero and two different types of RetainerAccessory \\
    \hline
    \color[HTML]{3531FF} \textbf{5} & The TotalPrice attribute of a SupplyOffer must be equal to the sum of the prices of the SnowstopProduct entities associated to it through the Offer-Prod relationship multiplied by their price\\
    \hline
  \end{tabular}
\end{table}

\pagebreak

\subsection{Reasons for the restructuring steps}
\begin{itemize}
  \item By carefully analysing the domain of the database, it follows that a SnowRetainer can only be of two types, namely Grid or Tube, we can avoid to represent the GridRetainer and TubeRetainer relations and instead add an Attribute to SnowRetainer with the constraint of it being either 'Grid' or 'Tube'. Instead of having the attributes 'Height' and 'Diameter' (which we have on GridRetainer and TubeRetainer) we can create a new attribute to represent them both and call it 'Measure', in addition we can also add the attribute 'Profile' (previously on GridRetainer) and make it optional, by also adding a constraint on it being null only if the attribute 'Type' has the value 'Tube'. In this way we avoid one write in operation 2b.
  \item By extending the same concept explained in the last point also to RetainerHolder and RetainerAccessory we can add the same 'Type' attribute to these two relations. In this way we can avoid creating the relationships H-Compatible and A-Compatible by modifying the external constraint nr 4 on the SupplyOffer to check that the Type attributes correspond. As a result, operation 2a will need respectively 25 and 225 less writes and in addition 50 and 450 less reads. Operation 5 will need 2 less reads and in addition the satisfaction of the external constraint on SupplyOffer will become easier to manage.
  \item We merge Customer with Loc-Customer relationship, by adding Zip and City attributes to the relation Customer. This avoids one write in operation 1.
  \item We merge Phone with Num-Customer relationship, by adding CustomerCode attribute to the relation Phone. Similarly as above, this decreases the number of writes needed for operation 1 by an average of 1.5.
  \item We merge City with Is-In-Prov relationship, by adding new attribute Province to the relation City, in this way reduce the numbers of joins needed to obtain the climatic zone of a city from 3 to 2.
  \item We merge SupplyOffer with Has-Computation relationship, by adding all the attributes of the relationship to the relation SupplyOffer. In this way we avoid creating an additional relationship when creating a SupplyOffer and we need to perform one less join operation when accessing a SnowloadComputation that is associated to a SupplyOffer.
  \item We merge BuildingSite with Comp-For-BS relationship, by adding the Code of SnowloadComputation relation to the  attributes of BuildingSite relation. Since the two relations are usually accessed together we can merge them in order to have one less write on operation 4.
\end{itemize}

\pagebreak

\subsection{Application load after the restructuring}

\subsubsection{Restructured table of volumes and operations}

\vspace{12px}

{\centering \textbf{Table of volumes after the restructuring of the relational schema}\\}

\begin{table}[H]
  \def\arraystretch{1.25}%  1 is the default, change whatever you need
  \centering
  \begin{tabular}{ | m{4.5cm} | m{4.5cm}| m{4.5cm} |}
    \hline
    {\textbf{\large Concept}} & {\textbf{\large Construct}} & {\textbf{\large Volume}} \\
    \hline
    \color[HTML]{3531FF} \textbf{SnowstopProduct} & Entity & 500  \\
    \hline
    \color[HTML]{3531FF} \textbf{SnowRetainer} & Entity & 50 \\
    \hline
    \color[HTML]{3531FF} \textbf{RetainerHolder} & Entity & 400 \\
    \hline
    \color[HTML]{3531FF} \textbf{RetainerAccessory} & Entity & 50 \\
    \hline
    \color[HTML]{3531FF} \textbf{SupplyOffer} & Entity & 5000\\
    \hline
    \color[HTML]{3531FF} \textbf{Customer} & Entity & 2500\\
    \hline
    \color[HTML]{3531FF} \textbf{Phone} & Entity & 3750 \\
    \hline
    \color[HTML]{3531FF} \textbf{SnowloadComputation} & Entity & 4000\\
    \hline
    \color[HTML]{3531FF} \textbf{BuildingSite} & Entity & 4000\\
    \hline
    \color[HTML]{3531FF} \textbf{City} & Entity & 8000 \\
    \hline
    \color[HTML]{3531FF} \textbf{Province} & Entity & 100 \\
    \hline
    \color[HTML]{3531FF} \textbf{Offer-Prod} & Relationship & 15000\\
    \hline
    \color[HTML]{3531FF} \textbf{For-Customer} & Relationship & 6000\\
    \hline
  \end{tabular}
\end{table}

\textbf{Operations of interest:}

\begin{enumerate}
  \item Insert a new customer.
  \item Insert a new snowstop product, defining also the type and the compatibility.
  \item Insert a new city.
  \item Create a snowload computation on a given building site.
  \item Create a supply offer.
  \item List all the supply offers made for a given customer.
  \item Update prices of snowstop products.
  \item Update zip code and name of cities.
\end{enumerate}

\vspace{12px}

{\centering \textbf{Table of Operations}\\}

\begin{table}[H]
  \def\arraystretch{1.25}%  1 is the default, change whatever you need
  \centering
  \begin{tabular}{ | m{2.5cm} | m{3.5cm}| m{3.5cm} |}
    \hline
    {\textbf{\large Operation}} & {\textbf{\large Type}} & {\textbf{\large Frequency}} \\
    \hline
    \color[HTML]{3531FF} \textbf{1} & Interactive & 20/day  \\
    \hline
    \color[HTML]{3531FF} \textbf{2} & Interactive & 10/month  \\
    \hline
    \color[HTML]{3531FF} \textbf{3} & Interactive & 5/day  \\
    \hline
    \color[HTML]{3531FF} \textbf{4} & Interactive & 40/day  \\
    \hline
    \color[HTML]{3531FF} \textbf{5} & Interactive & 50/day  \\
    \hline
    \color[HTML]{3531FF} \textbf{6} & Batch & 10/week  \\
    \hline
    \color[HTML]{3531FF} \textbf{7} & Interactive & 2/year  \\
    \hline
    \color[HTML]{3531FF} \textbf{8} & Batch & 1/month  \\
    \hline
  \end{tabular}
\end{table}

\pagebreak

\subsubsection{Restructured access tables}

In the total cost evaluation we assume that a write access costs like two read accesses.


\vspace{12px}

{\centering \textbf{Access table for Operation 1}\\}
\begin{table}[H]
  \def\arraystretch{1.10}%  1 is the default, change whatever you need
  \centering
  \begin{tabular}{ | m{4cm} | m{4cm}| m{3cm} | m{2cm} |}
    \hline
    {\textbf{\large Concept}} & {\textbf{\large Construct}} & {\textbf{\large Accesses}} & {\textbf{\large Type}} \\
    \hline
    \color[HTML]{3531FF} Customer & Entity & 1 & W \\
    \hline
    \color[HTML]{3531FF} Phone & Entity & 1.5* & W \\
    \hline
  \end{tabular}
  * \small{We assume that a customer has normally between 1 and 2 phone numbers, thus 1.5 on average}
\end{table}
Total: 2.5*20 write accesses = 100 accesses per day

\vspace{12px}

{\centering \textbf{Access table for Operation 2a (RetainerHolder or RetainerAccessory)}\\}
\begin{table}[H]
  \def\arraystretch{1.10}%  1 is the default, change whatever you need
  \centering
  \begin{tabular}{ | m{4cm} | m{4cm}| m{3cm} | m{2cm} |}
    \hline
    {\textbf{\large Concept}} & {\textbf{\large Construct}} & {\textbf{\large Accesses}} & {\textbf{\large Type}} \\
    \hline
    \color[HTML]{3531FF} SnowstopProduct & Entity & 1 & W \\
    \hline
    \color[HTML]{3531FF} RetainerHolder /\newline RetainerAccessory & Entity & 1 & W \\
    \hline
  \end{tabular}
\end{table}
Total: 2*10 write accesses = 40 accesses per month

\vspace{12px}

{\centering \textbf{Access table for Operation 2b (SnowRetainer)}\\}
\begin{table}[H]
  \def\arraystretch{1.10}%  1 is the default, change whatever you need
  \centering
  \begin{tabular}{ | m{4cm} | m{4cm}| m{3cm} | m{2cm} |}
    \hline
    {\textbf{\large Concept}} & {\textbf{\large Construct}} & {\textbf{\large Accesses}} & {\textbf{\large Type}} \\
    \hline
    \color[HTML]{3531FF} SnowstopProduct & Entity & 1 & W \\
    \hline
    \color[HTML]{3531FF} SnowRetainer & Entity & 1 & W \\
    \hline
  \end{tabular}
\end{table}
Total: 2*10 write accesses = 40 accesses per month

\vspace{12px}

{\centering \textbf{Access table for Operation 3}\\}
\begin{table}[H]
  \def\arraystretch{1.10}%  1 is the default, change whatever you need
  \centering
  \begin{tabular}{ | m{4cm} | m{4cm}| m{3cm} | m{2cm} |}
    \hline
    {\textbf{\large Concept}} & {\textbf{\large Construct}} & {\textbf{\large Accesses}} & {\textbf{\large Type}} \\
    \hline
    \color[HTML]{3531FF} City & Entity & 1 & W \\
    \hline
  \end{tabular}
\end{table}
Total: 1*5 write accesses = 10 accesses per day

\vspace{12px}

{\centering \textbf{Access table for Operation 4}\\}
\begin{table}[H]
  \def\arraystretch{1.10}%  1 is the default, change whatever you need
  \centering
  \begin{tabular}{ | m{4cm} | m{4cm}| m{3cm} | m{2cm} |}
    \hline
    {\textbf{\large Concept}} & {\textbf{\large Construct}} & {\textbf{\large Accesses}} & {\textbf{\large Type}} \\
    \hline
    \color[HTML]{3531FF} BuildingSite & Entity & 1 & W \\
    \hline
    \color[HTML]{3531FF} SnowloadComputation & Entity & 1 & W \\
    \hline
  \end{tabular}
\end{table}
Total: 2*40 write accesses = 160 accesses per day

\vspace{12px}

{\centering \textbf{Access table for Operation 5}\\}
\begin{table}[H]
  \def\arraystretch{1.10}%  1 is the default, change whatever you need
  \centering
  \begin{tabular}{ | m{4cm} | m{4cm}| m{3cm} | m{2cm} |}
    \hline
    {\textbf{\large Concept}} & {\textbf{\large Construct}} & {\textbf{\large Accesses}} & {\textbf{\large Type}} \\
    \hline
    \color[HTML]{3531FF} SnowloadComputation & Entity & 1 & R \\
    \hline
    \color[HTML]{3531FF} BuildingSite & Entity & 1 & R \\
    \hline
    \color[HTML]{3531FF} RetainerHolder & Entity & 1 & R \\
    \hline
    \color[HTML]{3531FF} SnowRetainer & Entity & 1 & R \\
    \hline
    \color[HTML]{3531FF} RetainerAccessory & Entity & 1 & R \\
    \hline
    \color[HTML]{3531FF} SnowstopProduct & Entity & 3 & R \\
    \hline
    \color[HTML]{3531FF} SupplyOffer & Entity & 1 & W \\
    \hline
    \color[HTML]{3531FF} Offer-Prod & Relationship & 3* & W \\
    \hline
    \color[HTML]{3531FF} For Customer & Relationship & 1 & W \\
    \hline
  \end{tabular}
  \small{* On average three products offered}
\end{table}
Total: 5*50 write accesses + 8*50 read accesses = 1000 accesses per day

\vspace{12px}

{\centering \textbf{Access table for Operation 6}\\}
\begin{table}[H]
  \def\arraystretch{1.10}%  1 is the default, change whatever you need
  \centering
  \begin{tabular}{ | m{4cm} | m{4cm}| m{3cm} | m{2cm} |}
    \hline
    {\textbf{\large Concept}} & {\textbf{\large Construct}} & {\textbf{\large Accesses}} & {\textbf{\large Type}} \\
    \hline
    \color[HTML]{3531FF} Customer & Entity & 1 & R \\
    \hline
    \color[HTML]{3531FF} For-Customer & Relation & 5 & R \\
    \hline
    \color[HTML]{3531FF} SupplyOffer & Entity & 5* & R \\
    \hline
  \end{tabular}
  \small{* We suppose on average 5 supply offers for customer}
\end{table}
Total: 11*10 read accesses = 110 read accesses per week

\vspace{12px}

{\centering \textbf{Access table for Operation 7}\\}
\begin{table}[H]
  \def\arraystretch{1.10}%  1 is the default, change whatever you need
  \centering
  \begin{tabular}{ | m{4cm} | m{4cm}| m{3cm} | m{2cm} |}
    \hline
    {\textbf{\large Concept}} & {\textbf{\large Construct}} & {\textbf{\large Accesses}} & {\textbf{\large Type}} \\
    \hline
    \color[HTML]{3531FF} SnowstopProduct & Entity & 500 & R \\
    \hline
    \color[HTML]{3531FF} SnowstopProduct & Entity & 500 & W \\
    \hline
  \end{tabular}
\end{table}
Total: 500*2 read accesses + 500*2 write accesses = 3.000 accesses per year

\vspace{12px}

{\centering \textbf{Access table for Operation 8}\\}
\begin{table}[H]
  \def\arraystretch{1.10}%  1 is the default, change whatever you need
  \centering
  \begin{tabular}{ | m{4cm} | m{4cm}| m{3cm} | m{2cm} |}
    \hline
    {\textbf{\large Concept}} & {\textbf{\large Construct}} & {\textbf{\large Accesses}} & {\textbf{\large Type}} \\
    \hline
    \color[HTML]{3531FF} City & Entity & 8000 & R \\
    \hline
    \color[HTML]{3531FF} City & Entity & 50* & W \\
    \hline
  \end{tabular}
  \small{* We assume only a small percentage of city have their zip code changed}
\end{table}
Total: 8000*1 read accesses + 50*1 write accesses = 8.100 accesses per month

\pagebreak

\vspace{12px}